\documentclass{standalone}
\begin{document}
\subsection{Aufgabe 13.11}
Die Multiplikation von $z \in \mathbb{C}$ mit der imaginären Einheit $i$ entspricht einer mathematisch positiven Rotation um 90 Grad in der komplexen Ebene.
Sie lässt sich umschreiben durch
$$ e^{i \frac{\pi}{2}} = \cos \frac{\pi}{2} + i \sin \frac{\pi}{2} = i $$
Ersetzt man alle Vorkommnisse von $z$ durch die exponentielle Schreibweise $z = k e^{i \phi}$, erhält man

$$ k = e^{i \frac{\pi}{2}} k^2e^{i 2 \phi} = e^{i (2 \phi + \frac{\pi}{2})} $$

Trennung in Real- und Imaginärteil, sowie Koeffizientenvergleich führen zu 

\begin{align}
    \begin{matrix} 
        k \\
        0
    \end{matrix}
    &=
    \begin{matrix} 
        k^2 \cos (2 \phi + \frac{pi}{2}) \\
        k \sin (2 \phi + \frac{pi}{2})
    \end{matrix} \\
    \begin{matrix} 
        \frac{1}{k} \\
        0
    \end{matrix}
    &=
    \begin{matrix} 
        \cos (2 \phi + \frac{pi}{2}) \\
        \sin (2 \phi + \frac{pi}{2}) \wedge k \neq 0
    \end{matrix} \\
    \begin{matrix} 
        \phi \\
        \phi \\
    \end{matrix}
    &=
    \begin{matrix} 
        \frac{\arccos \frac{1}{k}}{2} - \frac{\pi}{4} \\
        \frac{\arcsin (0)}{2} -\frac{\pi}{4} = \frac{n\pi}{2} -\frac{\pi}{4} \text{, für } n \in \mathbb{Z}
    \end{matrix} \\    
\end{align}

Gleichsetzen:
\begin{align}
    \arccos \frac{1}{k} &= n\pi \\
    k &= \cos (n\pi) = \pm 1
\end{align}

Die Gleichung gilt somit für alle $z:=ke^{i\phi}$ mit $k = \cos(n\pi)$ und $\phi = n\pi$.

Geometrisch interpretiert: da $\abs{z}$ stets reell ist, dürfen die Lösungen der Gleichungen keinen Imaginärteil mehr besitzen. Daraus ergibt sich $\phi = n\pi$ mit $n \in \mathbb{Z}$, da somit die Zahl in der komplexen Ebene immer auf der x-Achse liegt.
Da die Quadrierung der komplexen Zahl zu einer Quadrierung ihres Betrags führt, und der Betrag selbst stets positiv ist, muss $k = \pm 1$ sein.

\subsection{Aufgabe 13.12}

\begin{enumerate}[a)]
    \item $$ arg(z_1 z_2) = arg(z_1) + arg(z_2) + 2\pi k$$
    \begin{align}
        z_1 &= k_1 e^{i\phi_1} \\
        z_2 &= k_2 e^{i\phi_2} \\
        z_1 z_2 &= (k_1 k_2) e^{i (\phi_1 + \phi_2)} \\
        arg(z_1 z_2) &= \phi_1 + \phi_2
    \end{align}
    \item $$ arg(z_1 \overline{z_2}) = arg(z_1) - arg(z_2) + 2\pi k $$
    \begin{align}
        z_2 &= k_2 (\cos \phi_2 + i \sin \phi_2) \\
        \overline{z_2} &= k_2(\cos \phi_2 - i \sin \phi_2) \\
        &= k_2(\cos (-\phi_2) + i \sin (-\phi_2)) \\
        &= k_2e^{-i\phi_2} \\
        z_1 \overline{z_2} &= k_1 k_2 e^{\phi_1 - \phi_2} \\
        arg(z_1 \overline{z_2}) &= \phi_1 - \phi_2
    \end{align}
\end{enumerate}
Für beide Aufgabenteile gilt, aufgrund der Periodizität der komplexen Expontentialfunktion, dass Addition um $2 \pi k $ mit $k \in \mathbb{Z}$ den Wert nicht verändert.

Für welche $z_1,z_2$ gilt die erste Aussage?

Sie gilt für $\{z \in \mathbb{C} \mid \abs{z} \neq 0\}$. Die Multiplikation mit $0$ führt dazu, dass der Punkt auf den Koordinatenursprung gezogen wird. Da das Argument relativ zu den Koordinatenachsen gebildet wird, ist es im Koordinatenursprung undefiniert.


\end{document}

\documentclass{standalone}
\begin{document}
    
\section{Übung 5}

\subsection{Aufgabe 5.1}

\subsection{Aufgabe 5.2}
•Es gilt zu beweisen, dass 
\begin{align}
sin(x)*cos(x)= 1/2*sin(x-y)+1/2*sin(x+y)
\end{align}
gilt. Um die Aufgabe zu lösen verwenden wir die Additionstheoreme, die wie folgt lauten:
\begin{align}
sin(x+y)=sin(x)*cos(y)+cos(x)*sin(y)
\end{align}
\begin{align}
sin(x-y)=sin(x)*cos(y)-cos(x)*sin(y)
\end{align}
Gleichung (2) und (3) setzen wir nun jeweils für die rechte Seite der Gleichung (1) ein, klammern 1/2 aus und erhalten: 
\begin{align}
sin(x)*cos(x)= 1/2*(sin(y)*cos(y)-cos(x)*sin(y)+sin(x)*cos(y)+cos(x)*sin(y))
\end{align} 
Nun kann man kürzen und bekommt nach dem zusammenaddieren:
\begin{align}
sin(x)*cos(y)=sin(x)*cos(y),
\end{align}
was eine wahre Aussage ist. Somit ist der Beweis vollbracht.


\end{document}
\documentclass{standalone}
\begin{document}

\subsection{Aufgabe 6.4}
\begin{itemize}
	\item[a)]
	\begin{itemize}
		\item[i)]
		A= $\begin{pmatrix}
			2 & 2 \\
			2 & 2
		\end{pmatrix};
		B=\begin{pmatrix}
			3 & 3 \\
			3 & 3
		\end{pmatrix} \leftrightarrow
		AB=BA=\begin{pmatrix}
			12 & 12 \\
			12 & 12
		\end{pmatrix}$\\
Der Produkt von dem Spalten- und Zeilenvektoren sind gleich, selbst wenn die beiden Umgekehrt sind. 
		\item[ii)]
	$A=
	B=\begin{pmatrix}
		n & n \\
		n & n
	\end{pmatrix} \leftrightarrow
	AB=BA=\begin{pmatrix}
		2n^2 & 2n^2 \\
		2n^2 & 2n^2
	\end{pmatrix}$\\
Jeder Mitglied von der Matrix(-en) sind gleich. Deswegen die Produkten von jeder Vektor (Spalten/Zeilen und ihre umgekehrte Zustand) ist gleich $=n^2$ 
\item[iii)]
$A=\begin{pmatrix}
	a & b \\
	c & d
\end{pmatrix}; 
B=\begin{pmatrix}
	0 & 0 \\
	0 & 0
\end{pmatrix} \leftrightarrow
AB=BA=\begin{pmatrix}
	0 & 0 \\
	0 & 0
\end{pmatrix}$\\
Die Multiplikationsreihenfolge ist egal wenn man ein Matrix (A) mit die Nullmatrix (B) multipliziert.
	\end{itemize}
	\item[b)] Wir bestimmen C und D als:\\
	
	$C=\begin{pmatrix}
		a & b \\
		c & d
	\end{pmatrix}; 
	D=\begin{pmatrix}
		e & f \\
		g & h
	\end{pmatrix} \leftrightarrow CD= \begin{pmatrix}
	ae+bg & af+bh \\
	ce+dg & cf+dh
\end{pmatrix}\leftrightarrow DC= \begin{pmatrix}
ea+fc & eb+fd  \\
ga+hc & gb+hd
\end{pmatrix}$ und eine konkrete Beispiel:\\
$C=\begin{pmatrix}
	1 & 2 \\
	3 & 4
\end{pmatrix}; 
D=\begin{pmatrix}
	5 & 6 \\
	7 & 8
\end{pmatrix} \leftrightarrow CD= \begin{pmatrix}
	19 & 22 \\
	43 & 50
\end{pmatrix}\leftrightarrow DC= \begin{pmatrix}
	23 & 34  \\
	31 & 46
\end{pmatrix}$

\end{itemize}



\end{document}